% !TeX document-id = {23893128-e2d2-445f-8ab8-79b297fa2673}
% !TeX TXS-program:compile = txs:///pdflatex/[--shell-escape]
\documentclass[spanish,addpoints,answers,a4paper]{exam}
\usepackage[utf8]{inputenc}
\usepackage[T1]{fontenc}
\usepackage[spanish]{babel}
\usepackage{libertine}
\usepackage[shortlabels]{enumitem}
\usepackage[tmargin=2cm,rmargin=2cm,lmargin=2cm]{geometry}
\pagestyle{plain}

\usepackage{minted}
\renewcommand{\solutiontitle}{\noindent\textbf{Solución}\par\noindent}
\begin{document}

\begin{center}
	\scshape\bfseries\LARGE\color{blue}
	Lenguaje de programación C\\
	Primera lista de ejercicios
\end{center}

\begin{center}
	\fbox{\fbox{\parbox{5.5in}{\centering
				Responda las preguntas en los espacios provistos en las hojas de preguntas. Los argumentos y la claridad de las respuestas se considerarán en la puntuación final.
			}}}
\end{center}
\vspace{0.1in}
\makebox[\textwidth]{Nombres y apellidos:\enspace\hrulefill}
\vspace{0.2in}
\makebox[\textwidth]{Nombres y apellidos del instructor:\enspace\hrulefill}

\begin{questions}
\question ¿Qué es un error semántico? Dé un ejemplo en castellano y uno en lenguaje C.

\begin{solution}
Un error semántico es un error de significado. Por ejemplo, considere la siguiente oración: Despreciados derivados cantan verdemente. La sintaxis está bien porque los adjetivos, los sustantivos, los verbos y los adverbios están en los lugares correctos, pero la oración no significa nada. En C, cometes un error semántico cuando sigues las reglas de C correctamente pero con un final incorrecto. El ejemplo tiene uno de esos errores:

\begin{listing}[H]
\footnotesize
\inputminted{c}{exercise1.c}
\caption{Programa \texttt{stillbad.c} que presenta errores semánticos.}
\label{lst:1}
\end{listing}

\end{solution}

\question ¿Qué es un error de sintáxis? Dé un ejemplo en castellano y uno en lenguaje C.

\begin{solution}
Un error de sintaxis ocurre cuando no se sigue las reglas de C. Es análogo a un error gramatical en español. Por ejemplo, considere la siguiente frase: Los errores frustrar pueden. Esta oración usa palabras válidas en español, pero no sigue las reglas para el orden de las palabras, y de todas maneras no tiene las palabras correctas. Los errores de sintaxis C usan símbolos C válidos en los lugares incorrectos.

\begin{listing}[H]
\footnotesize
\inputminted{c}{exercise2.c}
\caption{Programa \texttt{nogood.c} que presenta errores de sintáxis.}
\label{lst:2}
\end{listing}
\
{\footnotesize
\begin{minted}{bash}
nogood.c:6:1: error: unterminated /* comment/* this program has several errors n = 5;
^nogood.c:5:22: error: expected ')'
int n, int n2, int n3;
^nogood.c:4:1: note: to match this '('
(^
nogood.c:3:9: error: function cannot return function type 'int (int, int, int)'
int main(void)        ^
3 errors generated.
\end{minted}
}
\end{solution}

\question \textbf{Ichabod Bodie Marfoode} ha preparado el siguiente programa y neceista su corrección, ayúdelo:

\begin{listing}[H]
\footnotesize
\inputminted{c}{exercise3.c}
\caption{Programa con errores.}
\label{lst:3}
\end{listing}

\begin{solution}
\begin{listing}[H]
\footnotesize
\inputminted{c}{exercise3a.c}
\caption{Example of a listing.}
\label{lst:3a}
\end{listing}
\end{solution}

\question Asumiendo que cada ejemplo es parte de un programa completo. ¿Qué imprimirá cada parte?

\begin{parts}
\part \mintinline{c}{printf("Baa Baa Black Sheep");}
\part \mintinline{c}{printf("Have you any woo?\n");}
\part \mintinline{c}{printf("Begone\no creature of lard");}
\part
\begin{minted}{c}
int num;
num=2;
printf("%d+%d=%d",num,num,num+num);
\end{minted}
\end{parts}

\begin{solution}
a
\end{solution}

\question ¿Cómo imprimiría los valores de palabras y líneas en la forma?

¿Había 3020 palabras y 350 líneas?

Aquí, 3020 y 350 representan valores para las dos variables.

\begin{solution}
	A
\end{solution}

\question Escriba un programa que use una llamada \textbf{printf()} para imprimir en pantalla su nombre y apellido en una línea; use otro programa que use una línea para cada uno.

\begin{solution}
	A
\end{solution}

\question Escriba un programa para imprimir su nombre y dirección.

\begin{solution}
	A
\end{solution}

\question Escriba un programa que escriba su edad en años a días. No se preocupe por las fracciones de años.

\begin{solution}
	A
\end{solution}

\question Escriba un programa que escriba:

For he's a jolly good fellow!

For he's a jolly good fellow!

For he's a jolly good fellow!

Which nobody can deny!

\begin{solution}
	A
\end{solution}

\question Escriba un programa que cree una variable entera llamada \textbf{toes}, que le asigne el valor de $10$ y que calcule cuánto vale el doble.

\begin{solution}
	A
\end{solution}

\end{questions}

\end{document}