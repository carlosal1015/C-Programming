% !TeX document-id = {21d95b9e-6149-4bf7-a0af-622fadedeb8b}
% !TeX TXS-program:compile = txs:///pdflatex/[--shell-escape]
\documentclass[spanish,addpoints,answers,a4paper]{exam}
\usepackage[T1]{fontenc}
\usepackage[spanish]{babel}
\usepackage{libertine}
\usepackage[tmargin=2cm,rmargin=2cm,lmargin=2cm]{geometry}
\usepackage{amsmath,bm}
\usepackage[shortlabels]{enumitem}
\usepackage{siunitx}
\usepackage{multicol}

\usepackage{graphicx}
\graphicspath{{../../../images/}}
\pagestyle{plain}

\usepackage{minted}
\usepackage{hyperref}
\renewcommand{\solutiontitle}{\noindent\textbf{Solución}\par\noindent}
\renewcommand\listingscaption{Listado}
\newcommand{\unmarkedfntext}[1]{%
	\begingroup
	\renewcommand\thefootnote{}\footnote{#1}%
	\addtocounter{footnote}{-1}%
	\endgroup
}
\spanishdatedel
\decimalpoint

\def\LOGOUNI{%
	\begin{picture}(0,0)\unitlength=1cm
	\put (-13.5,-2.5) {\includegraphics[width=2.8cm]{logouni}}
	\end{picture}
}

\def\LOGOCTIC{%
	\begin{picture}(0,0)\unitlength=1cm
	\put (-8,-1.2) {\includegraphics[height=1cm]{logocticblack}}
	\end{picture}
}

\newcommand{\mychar}{%
	\begingroup\normalfont
	\includegraphics[height=\fontcharht\font`\B]{Octocat.png}%
	\endgroup
}
\begin{document}

\begin{center}
	\sffamily\bfseries\scshape
	{\Large UNIVERSIDAD NACIONAL DE INGENIERÍA}\LOGOUNI\\
	Centro de Tecnologías de la Información y Comunicaciones\LOGOCTIC\\
\end{center}

\vspace{.8cm}

\begin{center}\sffamily\bfseries\large
	C/C++ PROGRAMMING LANGUAGE\\
	TEMA $\bm{4}$: OPERATORS, EXPRESSIONS, AND STATEMENTS 
\end{center}

\vspace{.5cm}
\noindent
\makebox[\textwidth]{Nombres y apellidos:\enspace\hrulefill}
\vspace{0.2in}
\makebox[\textwidth]{Nombres y apellidos del instructor:\enspace MSc. César Manuel Sebastián Díez Chirinos.\hfill}

\begin{questions}

\question  Si todas las variables son del tipo \mintinline{c}{int}, calcule:

\begin{parts}
\part \mintinline{c}{X=(2+3)*6}
\part \mintinline{c}{X=(12+6)/2*3;}
\part \mintinline{c}{Y=x=(2+3)/4;}
\part \mintinline{c}{Y=3+2*(x=7/2);}
\end{parts}

\question Corrija los errores:

\begin{minted}{c}
main()
{
int=1,
float n;
scanf();
printf("watch out! Here come a bunch of fractions!\n");s
while(i<30)
n=1/i;
printf("%f", n);
printf("That's all, folks!\n");
}
\end{minted}

\question Hacer un \mintinline{c}{min_sec} interactivo no es fácil. ¿Cómo se puede mejorar?

\begin{minted}{c}
#include<stdio.h>
#define S_TO_M 60
main()
{
int sec, min, left;
printf("This program convierte segundos a minutos y");
printf("segundos.\n");
printf("Solo debe ingresar los segundos.\n");
printf("Ingrese 0 para finalizar el programa.\n");
while(sec>0){
scanf("%d", &sec);
min=sec/S_TO_M;
left=sec%S_TO_M;
printf("%d sec is %d min, %d sec.\n", sec, min, left);
printf("Next input?\n");
}
printf("See you!\n");
}
\end{minted}

\question Escriba un programa que pregunte por un entero, y que imprima los enteros desde este número hasta $10$ más de este. (Si fuera $5$, sería desde $5$ hasta $15$).

\question Escriba un programa que solicite un decimal e imprima su cubo.

\question Use un while loop para convertir el tiempo en minutos a el tiempo en horas y minutos.

\question ¿Qué imprimirá este programa?

\begin{minted}{c}
#include<stdio.h>
#define FORMAT "%s is a string\n"
main()
{
int num=0;
printf(FORMAT, FORMAT);
printf("%d\n", ++num);
printf("%d\n", num++);
printf("%d\n", num--);
printf("%d\n", num);
}
\end{minted}

\question Cambie el programa \texttt{addemup.c} para calcular cuánto dinero ganaría en $20$ días, si recibe $1\$$ el primer día, $2\$$ el segundo, $3\$$ el tercero y así.

\question Escriba un programa que convierta sus días en semanas y días.

\question Construya sentencia que hagan lo siguiente:

\begin{enumerate}[a.]
	\item Incremente la variable \mintinline{c}{x} por 10.
	\item Incremente la variable \mintinline{c}{x} por 1.
	\item Asigne dos veces la suma de \mintinline{c}{a} y \mintinline{c}{b} a \mintinline{c}{c}.
	\item Asigne \mintinline{c}{a} más dos veces \mintinline{c}{b} a \mintinline{c}{c}.
\end{enumerate}

\end{questions}

\begin{flushright}\bfseries
Centro de Tecnologías de la Información y Comunicaciones (CTIC)\\[2mm]
\today\unmarkedfntext{Código disponible en \href{https://github.com/carlosal1015/C-Programming}{\mychar{}}.}
\end{flushright}

\end{document}