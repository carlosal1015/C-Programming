% !TeX document-id = {21d95b9e-6149-4bf7-a0af-622fadedeb8b}
% !TeX TXS-program:compile = txs:///pdflatex/[--shell-escape]
\documentclass[spanish,addpoints,answers,a4paper]{exam}
\usepackage[utf8]{inputenc}
\usepackage[T1]{fontenc}
\usepackage[spanish]{babel}
\usepackage{libertine}
\usepackage[tmargin=2cm,rmargin=2cm,lmargin=2cm]{geometry}
\usepackage{amsmath}
\usepackage[shortlabels]{enumitem}
\usepackage{siunitx}
\usepackage{multicol}
\usepackage{hyperref}

\pagestyle{plain}

\usepackage{minted}
\renewcommand{\solutiontitle}{\noindent\textbf{Solución}\par\noindent}
\renewcommand\listingscaption{Listado}
\newcommand{\unmarkedfntext}[1]{%
	\begingroup
	\renewcommand\thefootnote{}\footnote{#1}%
	\addtocounter{footnote}{-1}%
	\endgroup
}
\spanishdatedel
\decimalpoint
\begin{document}

\begin{center}
\scshape\bfseries\LARGE\color{blue}
Lenguaje de programación C\\
Segunda lista de ejercicios
\end{center}

\begin{center}
\fbox{\fbox{\parbox{5.5in}
{\centering
Responda las preguntas en los espacios provistos en las hojas de preguntas. Los argumentos y la claridad de las respuestas se considerarán en la puntuación final.
}}}
\end{center}
\vspace{0.1in}
\makebox[\textwidth]{Nombres y apellidos:\enspace\hrulefill}
\vspace{0.2in}
\makebox[\textwidth]{Nombres y apellidos del instructor:\enspace MSc. César Manuel Sebastián Díez Chirinos.\hfill}

\begin{questions}

\question ¿Qué tipo de datos usaría para cada uno de los siguientes tipos de datos?
\begin{parts}
\part La población del Río Frito.
\part El peso promedio de una pintura de Rembrandt.
\part La letra más común en este capítulo.
\part El número de veces que esta letra ocurre.
\end{parts}

\begin{solution}
\begin{multicols}{2}
\begin{enumerate}[(a)]

\item Imprimirá:

\begin{minted}{bash}
pc@CTIC:~$ Baa Baa Black Sheep
\end{minted}

\item Imprimirá:

\begin{minted}{bash}
pc@CTIC:~$ Have you any woo?
pc@CTIC:~$ 
\end{minted}

\end{enumerate}
\end{multicols}
\end{solution}

\question Virgila Ann Xenopod ha inventado un progrma cargado de errores. Corríjale sus errores:

\begin{minted}{c}
#include <stdio.h>
main{
float g;h;
float tax, rate;
g=e21;
tax=rate*g;
}
\end{minted}

\question Identifique el tipo de datos (usados en declaraciones de sentencias) y el formato específico \mintinline{c}{printf()} para cada constante:

\begin{table}[H]
	\centering
\begin{tabular}{|l|l|c|c|}
	\hline
	\multicolumn{2}{|c|}{\bfseries Constant} & \textbf{Type} & \textbf{Specifier}	\\
	\hline
	A	&	012 	& 	& 	\\
	\hline
	B 	&	2.9e05L	&	&	\\
	\hline
	C	&	's'		& 	&	\\
	\hline
	D	&	10000 	& 	&	\\
	\hline
	E	&	'$\backslash$n' &	&	\\
	\hline
	F	&	20.0f	&	&	\\
	\hline
	G	&	0x44	&	&	\\
	\hline
\end{tabular}
\end{table}

\question Corrija este programa silly.\qquad(El / en C significa división)

\begin{minted}{c}
main()	/ Este programa es perfecto/
{
cows, legs integer;
printf();
scanf();
cows=legs/4;
printf("Esto implica que hay %f cows",cows)
}
\end{minted}

\question Encuentre que hace su sistema con desbordamiento de enteros, desbordamiento de puntos flotantes y el opuesto de desbordamiento de puntos flotantes.

\question Escriba un programa que pregunte cómo ingresa un valor en código \textbf{ASCII}, como $66$, e imprima el carácter en código \textbf{ASCII}.

\question Escriba un programa que informe una alerta e imprima el siguiente texto:

Asustada por el sonido, Sally gritó: \textbf{``Por la gran calabaza, ¡Qué fue eso!''}

\question Escriba un programa que lea un número de punto flotante e imprima primero en notación decimal y luego en notación exponencial. Puede tener esta estructura:

La entrada es $21.290000$ o $2.129000\text{e}+001$.

\question Aproximadamente hay $3,156\times10^{7}$ segundos en un año. Escriba un programa que solicite su edad en años y visualice su equivalente en segundos.

\question Las masas de una molécula simple tiene unos $3,0\times10^{-23}$ gramos. Un cuarto de agua es unos $950$ gramos. Escriba un programa que solicite la cantidad de agua, en cuartos, y visualice el número de moléculas de agua en esa cantidad. 
\end{questions}
\begin{flushright}\bfseries
Centro de Tecnologías de la Información y Comunicaciones (CTIC)\\[2mm]
\today\unmarkedfntext{Los archivos fuente lo encuentra  \href{https://github.com/carlosal1015/C-Programming}{\textbf{aquí}.}}
\end{flushright}
\end{document}