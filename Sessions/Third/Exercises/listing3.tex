% !TeX document-id = {21d95b9e-6149-4bf7-a0af-622fadedeb8b}
% !TeX TXS-program:compile = txs:///pdflatex/[--shell-escape]
\documentclass[spanish,addpoints,answers,a4paper]{exam}
\usepackage[T1]{fontenc}
\usepackage[spanish]{babel}
\usepackage{libertine}
\usepackage[tmargin=2cm,rmargin=2cm,lmargin=2cm]{geometry}
\usepackage{amsmath,bm}
\usepackage[shortlabels]{enumitem}
\usepackage{siunitx}
\usepackage{multicol}

\usepackage{graphicx}
\graphicspath{{../../../images/}}
\pagestyle{plain}

\usepackage{minted}
\usepackage{hyperref}
\renewcommand{\solutiontitle}{\noindent\textbf{Solución}\par\noindent}
\renewcommand\listingscaption{Listado}
\newcommand{\unmarkedfntext}[1]{%
	\begingroup
	\renewcommand\thefootnote{}\footnote{#1}%
	\addtocounter{footnote}{-1}%
	\endgroup
}
\spanishdatedel
\decimalpoint

\def\LOGOUNI{%
	\begin{picture}(0,0)\unitlength=1cm
	\put (-13.5,-2.5) {\includegraphics[width=2.8cm]{logouni}}
	\end{picture}
}

\def\LOGOCTIC{%
	\begin{picture}(0,0)\unitlength=1cm
	\put (-8,-1.2) {\includegraphics[height=1cm]{logocticblack}}
	\end{picture}
}

\newcommand{\mychar}{%
	\begingroup\normalfont
	\includegraphics[height=\fontcharht\font`\B]{Octocat.png}%
	\endgroup
}
\begin{document}

\begin{center}
\sffamily\bfseries\scshape
{\Large UNIVERSIDAD NACIONAL DE INGENIERÍA}\LOGOUNI\\
Centro de Tecnologías de la Información y Comunicaciones\LOGOCTIC\\
\end{center}

\vspace{.8cm}

\begin{center}\sffamily\bfseries\large
C/C++ PROGRAMMING LANGUAGE\\
TEMA $\bm{3}$: CHARACTER STRINGS AND FORMATTED I/0
\end{center}

\vspace{.5cm}
\noindent
\makebox[\textwidth]{Nombres y apellidos:\enspace\hrulefill}
\vspace{0.2in}
\makebox[\textwidth]{Nombres y apellidos del instructor:\enspace MSc. César Manuel Sebastián Díez Chirinos.\hfill}

\begin{questions}

\question Escriba un programa que muestre cada flag visto en este capítulo.

\begin{solution}
En el programa \ref{lst:3.1} se muestra los cuatro tipos de \textit{flag}:

\begin{enumerate}
	\item En el primer \mintinline{c}{printf()}, se usa como \textit{flag} el signo $+$ o $-$ para alinear la cadena hacia la derecha o hacia la izquierda, respectivamente.
	
	\item En el segundo \mintinline{c}{printf()}, también se usa como \textit{flag} el signo $+$ o $-$ para mostrar el signo de un número flotante.%\footnote{Para el caso de números enteros, lo que hace el signo es mover tantos espacios como se indique.}
	
	\item En el tercer \mintinline{c}{printf()}, Use un formulario alternativo para la especificación de conversión. Produce un 0 inicial para la forma \% o y un 0x o 0X inicial para la forma \% x o\% X, respectivamente. Para todas las formas de punto flotante, \# garantiza que se imprima un carácter de punto decimal, incluso si no hay dígitos. Para los formularios \% g y \% G, evita que se eliminen los ceros finales.
	
	\item En el cuarto \mintinline{c}{printf()}, Para formularios numéricos, rellenar el ancho del campo con ceros a la izquierda en lugar de espacios. Esta bandera se ignora si a - bandera está presente o si, para una forma entera, se especifica una precisión.
\end{enumerate}

\begin{listing}[H]
	\footnotesize
	\inputminted{c}{exercise3_1.c}
	\caption{Programa \texttt{exercise3\_1.c}.}
	\label{lst:3.1}
\end{listing}
\end{solution}

\question Escriba un programa que nos dé la siguiente salida:

\begin{minted}{bash}
**42**42**-42**
**    6**  006**00006**  006
\end{minted}

\begin{solution}
\begin{listing}[H]
	\footnotesize
	\inputminted{c}{exercise3_2.c}
	\caption{Programa \texttt{exercise3\_2.c}.}
	\label{lst:3.2}
\end{listing}
\end{solution}

\question ¿Es factible la siguiente línea?

\begin{minted}{c}
rv = printf(%d F es el punto de ebullición del agua.");
\end{minted}

si es factible, dé un ejemplo.

\begin{solution}
Sí es factible, vea el listing~\ref{lst:3.3}:

\begin{listing}[H]
	\footnotesize
	\inputminted{c}{exercise3_3.c}
	\caption{Programa \texttt{exercise3\_3.c}.}
	\label{lst:3.3}
\end{listing}
\end{solution}

\question Escriba un programa donde se especifique el uso de \mintinline{bash}{*}:

\begin{parts}
	\part En la función \mintinline{c}{printf()}
	\part En la función \mintinline{c}{scanf()}
\end{parts}

\begin{solution}

\begin{parts}
\part En el listing~\ref{lst:3.4a} se muestra un programa que hace uso de \mintinline{bash}{*} en la función \mintinline{c}{printf()}:
	
\begin{listing}[H]
	\footnotesize
	\inputminted{c}{exercise3_4a.c}
	\caption{Programa \texttt{exercise3\_4a.c}.}
	\label{lst:3.4a}
\end{listing}

\part En el listing~\ref{lst:3.4b} se muestra un programa que hace uso de \mintinline{bash}{*} en la función \mintinline{c}{scanf()}:

\begin{listing}[H]
	\footnotesize
	\inputminted{c}{exercise3_4b.c}
	\caption{Programa \texttt{exercise3\_4b.c}.}
	\label{lst:3.4b}
\end{listing}
\end{parts}
\end{solution}

\question Suponga que un programa inicia con:

\begin{minted}{c}
#define BOOK "War and Peace"
main()
{
float cost=12.99;
float percent=80.0;
\end{minted}

Ahora construya una sentencia \mintinline{c}{printf()} que use \mintinline{c}{BOOK} y \mintinline{c}{cost} e imprima:

\begin{minted}{bash}
Esta copia de "War y Peace" se vende por $12.99.
Es decir, el 80% de la lista.
\end{minted}

\begin{solution}
\begin{listing}[H]
	\footnotesize
	\inputminted{c}{exercise3_5.c}
	\caption{Programa \texttt{exercise3\_5.c}.}
	\label{lst:3.5}
\end{listing}
\end{solution}

\question Asuma que cada uno de los siguientes ejemplos es parte de un programa completo, ¿qué aparecerá en cada una?

\begin{enumerate}[a.]
\item \mintinline{c}{printf("El vendió el cuadro por $%2.2f.\n", 2.345e2);}
\item \mintinline{c}{printf("%c%c%c\n",'H',105,'\41');}

\item \mintinline{c}{#define Q "Su Hamlet era divertido sin ser vulgar."}

\mintinline{c}{printf("%s\ntiene %d carácteres.\n", Q, strlen(Q));}

\item \mintinline{c}{printf("Es %2.2e los mismo que %2.2f?.\n",1201.0,1201.0);}
\end{enumerate}

\begin{solution}
Si creamos el listing~\ref{lst:3.6} con los ítemes de arriba y lo incluimos de manera adecuada aparecerá los siguientes mensajes:

\begin{listing}[H]
	\footnotesize
	\inputminted{c}{exercise3_6.c}
	\caption{Programa \texttt{exercise3\_6.c}.}
	\label{lst:3.6}
\end{listing}

\begin{enumerate}[a.]
\item \mintinline{bash}{El vendió el cuadro por $234.50.}

\item \mintinline{bash}{Hi!}

\item \begin{minted}{bash}
Su Hamlet era divertido sin ser vulgar.
tiene 39 carácteres.
\end{minted}

\item \mintinline{bash}{¿Es 1.20e+03 los mismo que 1201.00?.}

\end{enumerate}

\end{solution}

\question Escriba un programa que pregunte por su primer nombre, y luego su apellido, y luego los imprima primero el apellido y luego el nombre.

\begin{solution}
\begin{listing}[H]
	\footnotesize
	\inputminted{c}{exercise3_7.c}
	\caption{Programa \texttt{exercise3\_7.c}.}
	\label{lst:3.7}
\end{listing}
\end{solution}

\question Escriba un programa que requiera su altura en pulgadas y su nombre, y luego muestra la información en la siguiente forma:

\begin{minted}{bash}
Dabney, eres 6.028 pies de alto.
\end{minted}

\begin{solution}
\begin{listing}[H]
	\footnotesize
	\inputminted{c}{exercise3_8.c}
	\caption{Programa \texttt{exercise3\_8.c}.}
	\label{lst:3.8}
\end{listing}
\end{solution}

\question Escriba un programa que requiera el primer nombre de usuario y luego el apellido del usuario. Imprima los nombres en una línea y el número de letras de cada nombre en la siguiente línea.

\begin{solution}
\begin{listing}[H]
	\footnotesize
	\inputminted{c}{exercise3_9.c}
	\caption{Programa \texttt{exercise3\_9.c}.}
	\label{lst:3.9}
\end{listing}
\end{solution}

\question Repita el ejercicio anterior alineando cada nombre con cada número.

\begin{solution}
\begin{listing}[H]
	\footnotesize
	\inputminted{c}{exercise3_10.c}
	\caption{Programa \texttt{exercise3\_10.c}.}
	\label{lst:3.10}
\end{listing}	
\end{solution}

\end{questions}

\begin{flushright}\bfseries
Centro de Tecnologías de la Información y Comunicaciones (CTIC)\\[2mm]
\today\unmarkedfntext{Código disponible en \href{https://github.com/carlosal1015/C-Programming}{\mychar{}}.}
\end{flushright}

\end{document}