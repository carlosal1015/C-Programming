% !TeX document-id = {21d95b9e-6149-4bf7-a0af-622fadedeb8b}
% !TeX TXS-program:compile = txs:///pdflatex/[--shell-escape]
\documentclass[spanish,addpoints,answers,a4paper]{exam}
\usepackage[utf8]{inputenc}
\usepackage[T1]{fontenc}
\usepackage[spanish]{babel}
\usepackage{libertine}
\usepackage[tmargin=2cm,rmargin=2cm,lmargin=2cm]{geometry}
\usepackage{amsmath}
\usepackage[shortlabels]{enumitem}
\usepackage{siunitx}
\usepackage{multicol}
\usepackage{hyperref}

\pagestyle{plain}

\usepackage{minted}
\renewcommand{\solutiontitle}{\noindent\textbf{Solución}\par\noindent}
\renewcommand\listingscaption{Listado}
\newcommand{\unmarkedfntext}[1]{%
	\begingroup
	\renewcommand\thefootnote{}\footnote{#1}%
	\addtocounter{footnote}{-1}%
	\endgroup
}
\spanishdatedel
\decimalpoint
\begin{document}

\begin{center}
\scshape\bfseries\LARGE\color{blue}
Lenguaje de programación C\\
Tercera lista de ejercicios
\end{center}

\begin{center}
\fbox{\fbox{\parbox{5.5in}
{\centering
Responda las preguntas en los espacios provistos en las hojas de preguntas. Los argumentos y la claridad de las respuestas se considerarán en la puntuación final.
}}}
\end{center}
\vspace{0.1in}
\makebox[\textwidth]{Nombres y apellidos:\enspace\hrulefill}
\vspace{0.2in}
\makebox[\textwidth]{Nombres y apellidos del instructor:\enspace MSc. César Manuel Sebastián Díez Chirinos.\hfill}

\begin{questions}

\question Escriba un programa que muestre cada flag visto en este capítulo.

\question Escriba un programa que nos dé la siguiente salida:

\begin{minted}{bash}
**42**42**-42**
**    6**  006**00006**  006
\end{minted}

\question ¿Es factible la siguiente línea?

\begin{minted}{c}
rv = printf(%d F es el punto de ebullición del agua.");
\end{minted}

si es factible, dé un ejemplo.

\question Escriba un programa donde se especifique el uso de \inputminted{bash}{*}:

\question Construya sentencia que hagan lo siguiente:

\begin{parts}
	\part En la función \mintinline{c}{printf()}
	\part En la función \mintinline{c}{scanf()}
\end{parts}
\question Suponga que un programa inicia con:

\begin{minted}{c}
#define BOOK "War and Peace"
main()
{
float cost=12.99;
float percent=80.0;
\end{minted}

Ahora construya una sentencia \inputminted{c}{printf()} que use BOOK y cost e imprima:

\begin{flushleft}
Esta copia de ``War y Peace'' se vende por \$12.99.

Es decir, el 80\% de la lista.
\end{flushleft}

\question Asuma que cada uno de los siguientes ejemplos es parte de un programa completo, ¿qué aparecerá en cada una?
\begin{enumerate}[a.]
\item \inputminted{c}{printf("El vendió el cuadro por \$2.2f.\n", 2.345e2)};
\item \inputminted{c}{printf("\%c\%c\%c\n",'H','105','\41');}
\item \inputminted{c}{#define Q "Sy Hamlet era divertido sin ser vulgar."}
\item \inputminted{c}{printf("Es %2.2e los mismo que %2.2f?.\n",1201.0,1201.0);}
\end{enumerate}
\question Escriba un programa que pregunte por su primer nombre, y luego su apellido, y luego los imprima primero el apellido y luego el nombre.
\question Escriba un programa que requiera su peso en pulgadas y su nombre y que imprima:
\begin{minted}{bash}
Dabney, eres 6.028 pies de alto.
\end{minted}

\question Escriba un programa que requiera el primer nombre de usuario y luego el apellido del usuario. Imprima los nombres en una línea y el número de letras de cada nombre en la siguiente línea.

\question Repita el ejercicio anterior alineando cada nombre con cada número.

\end{questions}
\begin{flushright}\bfseries
Centro de Tecnologías de la Información y Comunicaciones (CTIC)\\[2mm]
\today\unmarkedfntext{Los archivos fuente lo encuentra  \href{https://github.com/carlosal1015/C-Programming}{\textbf{aquí}.}}
\end{flushright}
\end{document}